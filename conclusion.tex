\section{Conclusion}
\label{sec:Conclusion}
%[Be sure to end strong!  Tell the reader why your work is important.
%Explain the few key takeaways.  Benefits, eval results, usage, what is new.]
%
%[If code / data is available, reiterate]
%
%[optionally, explain future work]

A great deal of value can be gained from the analysis of an application's
activity.
% We set out to find a better way to pull this value out of the large
% volume of activity an activity produces
% Restate this sentence as the positive
Unfortunately,
the volume of activity
an application produces
makes it difficult
to separate out
unimportant sequences.
In this work,
we demonstrate how our new domain specific language,
PORT offers
a way to write concise and 
powerful
descriptions of
application activity sequences.
These descriptions
can be compiled
into programs that
both recognize the described activity
sequence
and modify its contents in order to
facilitate more active testing.
We used this capability to recreate the successful programs from earlier work on the SEA technique and showed that SEA can be extended to other activity representations, such as recorded USB traffic.

%We have also
%illustrated that PORT
%can be
%easily extended
%to support other activity
%representations.
%We used this capability to add support for USB Human Interface Device traffic so that PORT programs could be written to search and modify recordings of this activity.
%With these programs,
%we were able to both detect and simulate BADUSB-style attacks present in real world recordings of USB devices.
%Similar programs were able to simulate USB device identifier conflicts,
%a common source of USB-related bugs.

