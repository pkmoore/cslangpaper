\section{Our Motivating Example}
\label{SEC:background}

\textit{But it works on my machine!} goes the refrain of the tortured
software developer whose application works correctly when executed on a
development computer only to fail once it is deployed.  This occurs with
such frequency that the ``works on my machine'' phenomenon is a well known
source of pain
and frequent topic of discussion
in software and project management
literature~\cite{notreal}.
The problem is so widespread
that FAKESTUDY concluded
that \$XXX are spent annually on efforts to
recall,
fix,
and re-deploy applications
because of all the bugs
that slipped past extensive testing efforts
during development.

Earlier work has demonstrated that environmental bugs, those that occur due
to unanticipated qualities external to an application, are a major source
of such failures.  This fact continues to be reinforced by the regular
appearance of high impact environmental bugs in major pieces of
software~\cite{devzeroroot}.  And it appears that no class of application
is safe with environmental bugs affecting operating systems~\cite{bad},
user applications~\cite{bad} and even web applications~\cite{bad} in the
last year alone!

An initial thrust was made at this problem by Moore et al
when they employed
the Simulating Environmental Anomalies (SEA) technique
on applications' system calls~\cite{crashsim}.
This effort centered on the key insight
that problematic
environmental properties,
known as anomalies, are visible in the
communications between the components that make up an application.
They found that,
once captured,
these anomalies
could be
used to create simulations
that test
an application as if
it had encountered the captured anomalies
in the real world.
But system calls are just part of the story.
This paper documents an effort to augment SEA
so that its proven methodology
can be used to test a broader set of applications.
Our improvements
apply SEA's existing capability
to new bug domains
while maintaining its existing advantages and
dramatically
improving its reach.

