\section{Architecture}
\label{SEC:architecture}

... The key feature of SEA is the ability to capture and encode anomalies.

\preston{Anomalies can be described generically enough that they can be written
for one language and used to mutate another}

\begin{itemize}
  \item{Language describing anomalies}
  \item{Formal model of transducer that can implement anomalies}
  \item{Tool to compile description into said formal transducer}
  \item{Format-agnostic intermediate data structure (IDS) over which transducer
  operates}
  \item{Translation layer for converting to and from concrete data into IDS}
\end{itemize}

Here's the process:

\begin{enumerate}
\item{Anomaly is identified}
\item{Anomaly is described using CSlang}
\item{Anomaly is compiled into a transducer}
\item{Transducer is executed on a stream of incoming syscalls, rpc calls, etc}
\item{Mutated calls come out the other end}
\item{Application being tested is exposed to simulation of mutated calls}
\begin{itemize}
\item{This is CrashSimulator for system calls}
\item{Some other tool for RPC applications}
\end{itemize}
\item{Results come out the other side}
\preston{We need a collective term for system calls, rpc calls, etc}

\end{enumerate}
