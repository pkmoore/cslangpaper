\abstract{Earlier work has proven that information extracted from recordings of an
application’s activity can be tremendously valuable.
However, given the many requests  applications make to external entities, 
recognizing the handful of patterns that indicate the
potential for failure has been difficult at best. In this paper we propose a new technique for finding bugs that adapts 
event processing techniques
already used to locate patterns of interest in other domains. By creating PORT,
a new domain specific language, users can extract patterns in system call recordings and other streams,
and then rewrite input activity on the fly.
The former task can spot activity that indicates a bug,
while the latter produces a modified stream
for use in more active testing.
We employed PORT
to recreate the mutators and checkers
used by the SEA project
to modify and replay
the results of system calls. Our
 re-implementations achieved the same
efficacy and better reliability using fewer lines of
code.
We further illustrated PORT’s extensibility
by supporting remote
procedure call traces,
and showing how PORT can be used
to detect malicious application behavior.}