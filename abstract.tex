\begin{abstract}
Earlier work has proven that information extracted from recordings of an
application’s activity can be tremendously valuable.
However, during execution
applications make so many requests to external entities that
it is difficult to recognize the patterns that indicate the
potential for failure.
In this work, we propose addressing this problem by applying event processing techniques,
which are already used to identify patterns of interest in other domains.
To harness these techniques to identify bugs, we introduce PORT,
a new domain specific language, that allows users to construct programs to identify patterns in system call recordings and other streams,
and then rewrite the input activity on the fly.
The former task can spot activity that indicates a bug,
while the latter produces a modified stream
for use in more active testing.
We employed PORT
in recreating the mutators and checkers
used by the SEA project
to modify and replay
the results of system calls.
Our re-implementations achieved the same
efficacy and better reliability using fewer lines of
code.
We further illustrated PORT’s extensibility
by supporting remote
procedure call traces,
and showing how PORT can be used
to detect malicious application behavior.

\keywords{Firstkeyword \and Secondkeyword \and thirdkeyword}
\end{abstract}
