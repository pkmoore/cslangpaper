\begin{abstract}

Earlier work has demonstrated that there is tremendous value in
information that can be extracted from an application's activity.
The trick is in how to extract it.
Applications often make hundreds or thousands of requests
to external entities over the course of an execution
which makes it difficult
to recognize and respond to important patterns.
Existing event processing applications,
used successfully in fire alarms,
industrial control systems,
and computer network monitoring software,
have emerged as a potential solution
to this problem.
By design,
these systems
search
incoming streams of events
for a sequence,
and
react appropriately
when a pattern occurs.
Thus,
if application actions
can be processed like events,
this technology could be harnessed
as a tool for identifying bugs.
In this work,
we demonstrate
how
the creation of
a new domain specific language
called CSlang
allows users
to not only construct a program that describes an activity,
but also to rewrite an input activity stream on the fly.
The former may be used
to identify application activity
that may indicate the presence of a bug,
while the latter produces a modified stream
that may be used
for more active testing.
Using CSlang makes it easy
to identify sequences
in streams like event logs,
remote procedure call traces,
or system call activity.
For example, we employed CSlang
to recreate the ``anomalies''
used in
the SEA technique,
a project introduced in 2019
that was able to find bugs
by modifying and replaying the results of
an application's system calls.
In turn, we were also able to use
these anomalies
to test AAA applications
chosen from the Debian popularity contest
and found YYY new bugs in ZZZ
applications.
%%%
%Further, in order to illustrate CSlang's flexibility,
%we produced programs that can
%transform
%logged network traffic into firewall rules.
%These programs were able to identify common network attacks
%and block offending IP addresses.
%%%%
%Finally,
We further utilized this mechanism
to create programs that
could generate test cases
by examining and rewriting the contents of
remote procedure calls.
Using these programs we improved the test suites
of YYY applications identifying ZZZ new bugs in the process
and improving code coverage by AA\%.

\end{abstract}
