\begin{abstract}

  Earlier work has demonstrated that there is tremendous value in
  information that can be extracted from an application's activity.
  The trick is in how to extract it.
  Applications often make hundreds or thousands of requests
  to external entities over the course of an execution
  which makes recognizing important patterns difficult.
  We looked to existing event processing applications
  that are designed to search
  incoming streams of events
  for a sequence,
  and
  to react appropriately
  when such it occur as a potential solution to this problem.
  These programs have proven successful
  in fire alarms,
  industrial control systems,
  and computer network monitoring software.
  In this work,
  we illustrate how
  the principles behind event processing applications
  can be applied to identifying bugs
  if application actions
  are treated as events.
  To do so we created CSlang,
  a new domain specific language
  for describing
  sequences of application activity
  so that they may be identified
  in streams like event logs,
  remote procedure call traces,
  or system call activity.
  CSlang's key feature
  is the ease with which users
  can construct a program
  that not only describes an activity,
  but also  can
  rewrite an input activity stream on the fly
  producing a modified stream that can be used
  for more active testing.
  % suggest new inputs for test cases
  % offer correct API usage?+
  For example, we employed CSlang
  to recreate the ``anomalies''
  used in
  the SEA technique,
  a project introduced in 2019
  that was able to find bugs
  by modifying and replaying the results of
  an application's system calls.
  In turn, we were able to use
  these anomalies
  to test AAA applications
  chosen from the Debian popularity contest
  and found YYY new bugs in ZZZ
  applications.
  %%%
  %Further, in order to illustrate CSlang's flexibility,
  %we produced programs that can
  %transform
  %logged network traffic into firewall rules.
  %These programs were able to identify common network attacks
  %and block offending IP addresses.
  %%%%
  %Finally,
  We further utilized this mechanism
  to create programs that
  could generate test cases
  by examining and rewriting the contents of
  remote procedure calls.
  Using these programs we improved the test suites
  of YYY applications.  This effort identified ZZZ new bugs
  in these applications and improved code coverage by AA\%.



\end{abstract}
