\begin{abstract}

The failure of applications after deployment
is a constant source of headache and expense
for development teams.
It has been established
that a many of these failures
are caused by unanticipated anomalies that
appear in the communication between an application
and its environment.
Earlier work has shown that the
Simulating Environmental Anomalies (SEA) technique
is able to identify
bugs that are visible in the
results and side effects
of the system calls an application makes.
This work expands upon SEA's success by
providing a way to catch bugs that
are similarly visible in the communications
between application components.
This is accomplished by using our new
programming language,
cslang,
to describe a transducer
that can mutate valid communications
between application components
so that a chosen anomaly is inserted.
Then, in keeping with SEA's process,
This output feeds a testing process wherein
an application is exposed to a simulation of the
anomaly in order to get an idea of how it might respond
to it.
Using this new capability we evaluated AAA applications chosen
based on their popularity in the Debian popularity contest.
Our tests identified YYY new bugs in ZZZ applications.


\end{abstract}
