\begin{abstract}

  Event processing applications
  are able to examine incoming streams
  of events in order to
  recognize and react
  when important sequences occur.
  Such programs have proven successful
  in complex scenarios such as fire alarms,
  industrial control systems,
  and computer network monitoring software.
  In this work, we illustrate how similar techniques
  can be applied to the low level actions an application
  takes in order to identify bugs in the application.
  Central to this effort is CSlang,
  a new domain specific language
  for describing
  sequences of application activity
  so that they can be identified
  in streams like event logs,
  remote procedure call traces,
  or system call activity.
  CSlang's key feature
  is the ease with which users
  can construct a program
  describing important application activity
  and what action to take when it appears.
  This is accomplished by first ingesting
  the incoming activity stream
  and transforming it into a generic representation.
  The stream is then processed by an enhanced finite-state transducer
  created by compiling a CSlang program.
  Similar to a standard transducer,
  this model both is able to recognize the desired activity sequence
  and produce appropriate output when it occurs.
  In this case, the output is a copy of the input stream that has been
  modified as required by the CSlang program.
  Once the input stream has been consumed,
  the output is transformed back into
  the input's original format.
  %%%
  We employed CSlang
  in recreating the ``anomalies''
  used in
  the SEA technique,
  a project introduced in 2019,
  that showed success at finding bugs
  by modifying and replaying the results of
  an application's system calls.
  We then used these anomalies
  to test AAA applications
  chosen from the Debian popularity contest
  and found YYY new bugs in ZZZ
  applications.
  %%%
  %Further, in order to illustrate CSlang's flexibility,
  %we produced programs that can
  %transform
  %logged network traffic into firewall rules.
  %These programs were able to identify common network attacks
  %and block offending IP addresses.
  %%%%
  %Finally,
  Further,
  we created programs that could examine
  the contents of remote procedure calls,
  and modify them in order to generate test cases.
  Using these programs we improved the test suites
  of YYY applications.  This effort identified ZZZ new bugs
  in these applications and improved code coverage by AA\%.



\end{abstract}
