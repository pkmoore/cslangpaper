\begin{abstract}

  This work presents CSlang,
  a new domain specific language
  for describing and recognizing
  sequences of application activity
  so that they can be identified
  in streams like event logs,
  remote procedure call traces,
  or system call activity.
  CSlang's key feature
  is the ease with which users
  can construct programs
  capable of
  recognizing important application activity
  and taking action
  when it appears.
  %%%
  This is accomplished by first ingesting
  the incoming activity stream
  and transforming it into a generic representation.
  The stream is then processed by an enhanced finite-state transducer
  created by compiling a CSlang program.
  Similar to a standard transducer,
  this model both recognizes the activity sequence described in its
  CSlang program and outputs a modified stream based on its transition
  relation.
  Once the stream has been consumed, the output is transformed back into
  its original format.
  %%%
  We employed CSlang
  in recreating the ``anomalies''
  used in
  the SEA technique,
  a project introduced in 2019,
  that showed success at finding bugs
  by modifying and replaying the results of
  an application's system calls.
  We then used these anomalies
  to test AAA applications
  chosen from the Debian popularity contest
  and found YYY new bugs in ZZZ
  applications.
  %%%
  Further, in order to illustrate CSlang's flexibility,
  we produced programs that can
  transform logged network traffic into firewall rules.
  These programs were able to identify common network attacks
  and block offending IP addresses.
  %%%
  Finally,
  we created programs that could examine
  the contents of remote procedure calls,
  and modify them in order to generate test cases.
  Using these programs we improved the test suites
  of YYY applications.  This effort identified ZZZ new bugs
  in these applications and improved code coverage by AA\%.



\end{abstract}
