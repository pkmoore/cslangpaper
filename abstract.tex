Earlier work has demonstrated
the tremendous value
available in
information extracted
from recordings
of an application's activity.

However, applications often make numerous requests
to external entities during execution
which makes it difficult
to recognize and respond to important patterns.

We propose that event processing techniques,
which are capable
of identifying important patterns
in other domains,
are a solution
to this problem.

This work demonstrates
CSlang,
our new domain specific language
which harnesses
these techniques
to identify bugs.

CSlang allows users
to construct a program that can identify
patterns in streams like
system call recordings
and also to rewrite the input activity stream on the fly.

The former can
to identify activity
that indicates a bug,
while the latter produces a modified stream
for use in more active testing.

We employed CSlang
in recreating the mutators and checkers
used by
the SEA project
found bugs
by modifying and replaying the results of
system calls.

Our re-implementations
achieved the same efficacy
and better reliability
while consisting of XX\% less lines of code.

We illustrated CSlang's extensibility
by supporting remote procedure call traces,
and showing how CSlang can be used to
to detect malicious application behavior

