\begin{abstract}
Earlier work has proven that  information extracted from recordings of an
application’s activity can be tremendously valuable.
However, applications make so many requests to external entities during execution that
it is difficult to recognize and respond to important patterns. We
propose that event processing techniques, which are capable of
identifying patterns of interest in other domains, are a solution to
this problem. This work demonstrates how CSlang, our new domain
specific language can harness these techniques to identify
bugs. CSlang allows users to construct programs that can
identify patterns in system call recordings and other streams, and then
 rewrite the input activity on the fly. The former task
can spot activity that indicates a bug, while the latter
produces a modified stream for use in more active testing. We
employed CSlang in recreating the mutators and checkers used
by the SEA project to modify and replay the
results of system calls. Our re-implementations achieved the same
efficacy and better reliability using fewer lines of
code. We illustrated CSlang’s extensibility by supporting remote
procedure call traces, and showing how CSlang can be used to
to detect malicious application behavior.
\end{abstract}
