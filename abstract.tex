\begin{abstract}

Earlier work has demonstrated that,
when evaluating the quality of a piece of software,
there is tremendous value in
information that can be extracted
from recordings
of an application's activity.
The trick is in how to extract it.
Applications often make hundreds or thousands of requests
to external entities over the course of an execution
which makes it difficult
to recognize and respond to important patterns.
Existing event processing applications,
used successfully in fire alarms,
industrial control systems,
and computer network monitoring software,
have emerged as a potential solution
to this problem.
By design,
these systems
search
incoming streams of events
for a sequence,
and
react appropriately
when a pattern occurs.
Thus,
if application actions
can be processed like events,
this technology could be harnessed
as a tool for identifying bugs.
In this work,
we demonstrate
how
the creation of
a new domain specific language
called CSlang
allows users
to not only construct a program that describes an activity,
but also to rewrite an input activity stream on the fly.
The former may be used
to identify application activity
that may indicate the presence of a bug,
while the latter produces a modified stream
that may be used
for more active testing.
Using CSlang makes it easy
to identify sequences
in streams like event logs,
remote procedure call traces,
or system call activity.
To see how our efforts compared against other work
We employed CSlang
in recreating the mutators and checkers
used by
the SEA technique,
a project introduced in 2019
that was able to find bugs
by modifying and replaying the results of
an application's system calls.
We found our re-implementations
achieved the same efficacy
and better reliability
while consisting of XX\% less lines of code.
We further illustrated CSlang's extensibility
by adding support for remote procedure call traces
and creating programs that
could examine and rewrite the contents of
these recordings.

\end{abstract}
