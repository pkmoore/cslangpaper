\begin{abstract}

Earlier work in software testing has demonstrated
the value of
information that can be extracted
from recording
an application's activity.
The trick is in how to extract it.
The number
of requests
to external entities
made over the course
of an execution
makes it difficult
complicates the task
of recognizing and responding
to important patterns.
Existing event processing applications,
used successfully in
industrial control systems,
and computer network monitoring software,
have emerged as a potential solution
to this problem.
By design,
these systems
search
incoming streams of events
for a sequence,
and
react appropriately
when a pattern occurs.
Thus,
if application actions
can be processed like events,
this technology could be harnessed
as a tool for identifying bugs.
In this work,
we introduce
a new domain specific language
called CSlang
that can
not only construct a program
to describe an activity,
but can also
rewrite an input activity stream on the fly.
The former can
identify sequences in streams
like event logs,
remote procedure call traces,
or system call activity
that points
towards potential bugs
while the latter produces a modified stream
that can be applied to more active testing.
We employed CSlang
to recreate the mutators and checkers
used by
the SEA technique,
a 2019 project that demonstrated
bugs could be found
by modifying and replaying the results of
an application's system calls.
Our re-implementations
achieved the same efficacy
and better reliability
while consisting of XX\% less lines of code.
We further illustrated CSlang's extensibility
by adding support for remote procedure call traces
and creating programs that
could examine and rewrite the contents of
these recordings.

\end{abstract}
