\section{Related Work}
\label{SEC:related-work}

CSlang is a declarative domain-specific language designed to
assist developers in creating
applications that need to pattern match and perform
transformations on a single stream of possibly infinite events.
Potential applications
of CSlang could include
processing streams of system calls, RPC invocations or
web-browser events for the purpose of intrusion detection, fault-injection, or
other program-level testing.
In developing the language and its potential uses,
we examine
earlier work system call processing applications,
and event stream processing languages and algorithms.

\subsection{System Call Stream Processing Applications}
The
initial motivation for the development of CSlang was to create a tool to help
developers express match and transformation rules for system call patterns in
CrashSimulator~\cite{DBLP:conf/issre/MooreCFW19}.

There is a rich history of system call pattern
matching in the area of intrusion detection.  Broadly speaking, there are two
different types of system call based intrusion detection systems - misuse
intrusion detection and anomaly are two. In misuse intrusion
detection known patterns of application-specific malicious system call
sequences(intrusion signatures) are explicitly searched
for.~\cite{GARCIATEODORO200918}
In anomaly
detection it is assumed that the intrusion signatures are two, but deviation
from “normally observed” system call sequences are flagged as
malicious~\cite{DBLP:conf/sp/ForrestHSL96}.

As misuse-based intrusion detection systems are only
effective against previously known intrusion signatures they are rarely used in
isolation and are typically combined with anomaly-based systems.
Intrusion
detection systems also vary in the way in which they examine the system call
stream.
Some examine each system call in isolation,
without regard to previous
execution context,
looking for specific arguments that may imply malicious
intent.
Others look for fixed patterns or sequences of system calls executed
within a certain window of instructions.

Forrest et al.~\cite{DBLP:conf/sp/ForrestHSL96} describe
an anomaly detection system based on the premise that small deviations from
an application's previously exhibited system call execution pattern are likely
malicious. An application to be protected using this mechanism is first
“exercised” on various inputs. This training phase is intended to expose the
sequence of system calls found in valid uses of the application.
Using a sliding
window on the system call stream,
each witnessed pattern of six (this number can
be changed) contiguous system calls are cataloged in a database. After the
training phase the application's system call stream
is monitored and any detected
deviation from the expected system call patterns (stored in the database) are
acted upon in accordance with a predefined security policy.
A hidden Markov model
based implementation of this same system
is described in~\cite{DBLP:conf/sp/WarrenderFP99}.

Sekar et al.\cite{DBLP:conf/sp/SekarBDB01} extend the work
of~\cite{DBLP:conf/sp/ForrestHSL96} by proposing
an algorithm and associated optimizations
to create a finite state automaton that
learns the valid system call sequences of an application.
The automaton is not
limited to recognizing the small system call sequence sizes proposed in
~\cite{DBLP:conf/sp/ForrestHSL96}.
Instead,
the automaton learns the entire
sequence of system calls produced by each
run of the application.
To help minimize the size
of the automaton
the authors incorporate program counter information to recognize
loops.
In addition,
system calls made within standard libraries (such as libc)
are excluded from the automaton
as the authors felt that these system calls do not
necessarily help capture the unique nature of system call behavior within the
program.

Ko et al.~\cite{DBLP:conf/acsac/KoFL94} describe a real-time intrusion detection system
based on simple predicate logic and regular expressions.Each system call in the
steam is converted to a standard audit-policy record format which is then
matched against program policy. For example, the rule ‘exec “/bin/(sh | csh)” ’
allows a new shell to be invoked and the rule“~read(‘/etc/passed’)” prevents the
password file from being read. The audit-policy can only be applied to one
system call at a time - rules to recognize specific chains of system calls are
not supported.

Systrace~\cite{DBLP:conf/uss/Provos03} supports fine-grained process confinement
via system call monitoring and an associated policy language.The policy language
consists of a set of rules and triggers.
Triggers are executed once a rule
evaluates to true. Rules are checked against only one system call at a time.
Systrace intercepts each system call prior to execution, therefore this system
can also be used for fault injection in various testing scenarios.

Similar to system calls, remote procedure calls can also be abused for malicious
intent. Examining and acting upon a remote procedure call stream can help
mitigate or prevent such attacks. The remote procedure call protection mechanism
described in~\cite{DBLP:conf/uss/GiffinJM02} uses a finite-state machine and push-down automata
to model the possible valid remote call streams that can be generated in a
particular application. The model is created by static binary analysis
and modification of the application to be protected. The application's incoming
remote procedure call stream is vetted against the previously created model to
determine whether particular calls are valid and therefore
executable.

Phoebe~\cite{DBLP:journals/corr/abs-2006-04444} is a fault-injection framework that examines
the system call stream of an application to determine which system calls fail
naturally (during normal program execution). Based on this failure pattern
Phoebe generates fault-injection experiments designed to test the reliability of
an application when a failure occurs. These experiments are meant to magnify the
frequency of rarely occurring system call failures and to introduce them at
points where they had not previously occurred within the application being
tested. Only single system call matching is done - there is no facility to create
more elaborate fault-injection tests that are triggered after a specific
sequence of system calls are seen.

Christakis, et al.~\cite{DBLP:conf/icse/ChristakisEG017} describe a domain-specific language and runtime to allow developers to intercept and modify a Windows applications’ dynamic link library function calls (which includes system calls). The domain-specific language provides a mechanism to identify which function call or, by using pattern matching, set of function calls, should be intercepted by the runtime. Once intercepted, code written in the domain-specific language (which, in part, is a subset of the C programming language) can be executed. This system allows one to test for and simulate application errors in a language agnostic manner as the runtime operates at the level of the application binary. Note that the domain-specific language is only meant to match and intercept single function calls and was not designed to only trigger a code injection upon matching a specific sequence of function calls in a call stream. 

\subsection{Event Stream Processing Languages and Algorithms}

A streaming application
is a computer program
that consumes and
produces
a sequence of messages that each document that action has been performed.
A stream processing language is a domain-specific language
designed for expressing streaming applications~\cite{DBLP:journals/sigmod/HirzelBBVSV18}. CSlang clearly
fits within the definition of a streaming processing language.
Therefore, in this
section we look at work related to stream processing languages.

Pattern matching
over events streams is a paradigm
where it is decided if a stream of continuously arriving events
contains a particular sequence by evaluating whether or not
the events meet the requirements of a set of rules describing
the pattern being matched.
The events that meet these requirements are
transformed into new events for output.
Languages for pattern matching over event
streams are significantly richer than languages for regular expression
matching~\cite{DBLP:conf/sigmod/AgrawalDGI08}.
Stream processing languages typically provide automatic
support for naming, type checking, filtering, aggregating, classifying and
annotation of incoming events - providing many benefits over the traditional
stream-based text processing languages such as sed~\cite{Mcmahon1979sed} and
awk~\cite{DBLP:journals/spe/AhoKW79}.

Although CSlang is a stream processing language it does not
require all of the features provided by many stream processing systems. Features
such as combining multiple stream sources, out of order data retrieval, methods
for handling data loss, time window based event aggregation, database
integration, etc. are found in a number of modern stream processing
systems~\cite{DBLP:journals/csur/DayarathnaP18}.   Rather CSlang seems to fit within the special case of
stream processing known as complex event processing (CEP). CEP terminology refers
to data items in input streams as raw events and to data items in output streams
as composite (or derived) events. A CEP system uses patterns to inspect
sequences of raw events and then generates a composite event for each
match~\cite{DBLP:journals/ibmrd/HirzelAGJKKMNSSW13}.

Queries and transforms written for CEP systems are
typically compiled to a “systems” language (C, C++, etc.) to allow for fast
processing of the stream. During the compilation process automata are typically
built to recognize the patterns specified by the queries. Agrawal et
al.~\cite{DBLP:conf/sigmod/AgrawalDGI08} describes how patterns written in the SASE+ stream
processing language are converted to non-deterministic finite automata and
optimized.

MatchRegex~\cite{DBLP:conf/debs/Hirzel12} is a CEP engine for IBM’s Stream Processing
Language. Predicates defined on the various elements(raw events) in the event
stream can be utilized in the regular expression-based pattern matching
engine.MatchRegex supports regular expression operators such as “Kleene star”
and “Kleene plus” over patterns consisting of predicates (boolean expressions).
Standard regular expression semantics apply over these predicate-based patterns.
The non-deterministic finite automata generated from these patterns are free of
epsilon-transitions.
