\section{Introduction}
\label{SEC:introduction}

\textit{Clever quote here}

With the explosion of...

...applications are no longer single units sitting on a specific machine.

In fake study 1, researchers estimate that XX\% of distributed applications
contain bugs...


%No matter how well an application is tested before its release,
%new bugs always seem to emerge after deployment.
%In fact, Oracle estimates that 40\% of deployed applications
%contain critical defects -- a fact that is compounded that deployment
%increases the cost to fix these flaws by 100 times~\cite{OracleAppQuality}.
%One reason for this behavior
%is that these applications will operate within a diverse set of
%deployment \emph{environments},
%and variations between these environments tend to
%reveal previously undiscovered flaws.
%These flaws emerge from
%such factors as
%operating system APIs changing across versions
%~\cite{LinuxGlibcChanges},
%or small variations in file systems exhibiting subtle but critical
%differences~\cite{EXT4Layout, AppleHFS}.
%Even if the network and adapter are identical,
%network behavior can still diverge from what is expected~\cite{vbox},
%and these environmental differences greatly exacerbate
%the chance that an application will function incorrectly when deployed.

...something something something applications made of multiple
intercommunicating modules.


...This is increasingly true today's applications that typically consist of
multiple, intercommunicating modules


...These complex bugs are often difficult to reason about...


%These unforeseeable bugs
%complicate the work of application developers who, according to a
%recent survey conducted by ClusterHQ~\cite{ClusterHQSurvey},
%spend a significant portion of their time
%debugging errors that only appear in production.
%Numerous efforts have been made to reduce this burden.
%One approach
%is to hide environmental differences behind standard interfaces.
%Unfortunately,
%even specialized ``Write-Once, Run Anywhere'' environments
%that attempt to hide these differences,
%such as the Java Runtime Environment,
%are not perfect,
%leading them to be rechristened ``Write-Once, Debug Everywhere''~\cite{WODE}.
%A more direct approach would be
%to identify and fix deficiencies before deployment,
%but history has shown that,
%even if enormous effort is put forward,
%it may be insufficient to uncover these bugs.
%Microsoft employs thousands of engineers with nearly a
%1:1 ratio of testers to developers~\cite{Page2009}.
%Yet, a recent Windows Update released
%in response to the Spectre Intel CPU vulnerability
%resulted in machines with certain hardware configurations
%being rendered unbootable~\cite{kb4056892}.


%twe introduced Simulating Environmental Anomalies (SEA)
%and its key insight that
%problematic environmental properties,
%which we refer to as \textit{anomalies},
%can often be detected
%in the function calls,
%system calls,
%or other communications an application makes within an environment.
%When employing SEA,
%the anomalies
%unique to a given environment
%can be inserted into
%the communications of an application under test
%in such a way
%that its responses will indicate potential
%for failures upon deployment.
%These anomalies can be preserved and cataloged
%allowing them to be reused to test other applications,
%without requiring
%per-application effort.
%Over time,
%a corpus of environmental anomalies can be assembled
%that can help ensure
%future applications do not suffer from bugs of the past.


In earlier work, we made an initial thrust at this problem by employing the
Simulating Environmental Anomalies (SEA) technique on the system calls an
application makes. Our effort centered on the key insight, that problematic
environmental properties, known as anomalies, are visible in the
communications between the components that make up an application.  The
work documented in this paper augments SEA so that its proven methodology
can be used to test a broader set of applications.  Our improvements take
SEA beyond system calls by providing by providing a more generic way to
encode and simulate anomalies across the set of applications that use
structured communications like remote procedure calls.
This allows anomalies that were
valuable in testing one application to be directly used to test other
applications with similar communication strategies.  In effect, this takes
SEA's capability at finding environmental bugs and applies it to other bug
domains while both maintaining existing advantages of SEA and
dramatically improving its reach.




In order to evaluate our improvements to SEA, we needed a way to express
anomalies that was not tied to a particular domain.  To this end, we
implemented a new programming language, CSLang.  CSLang is a simple, but
powerful, language for describing the transducers needed to both identify
simulation opportunities and carry out the transformations required to make
said simulations a reality.  Initially, we used CSLang to implement the
mutators previously described in \preston{however you cite the first paper
here}. We found these mutators outperformed in terms of SOME METRIC HERE and
in XXX\% fewer lines of code.  To further our evaluation, we implemented ZZZ new
mutators capable of simulating anomalies in applications that use either
JSON-RPC or XML-RPC.  We employed these mutators in testing suitable
applications that ranked highly on Debian's popularity contest.  This
exercise identified AAA new bugs in the tested applications.  In the
future, these mutators can be used on similar application to expose more
bugs without additional test writing effort.

%In addition to demonstrating that
%CrashSimulator could find bugs, we were also able to
%show that developers with varying backgrounds
%could use the tool
%on real world applications with ease.
%We conducted a user study with
%12 undergraduate and graduate computer science students
%who were asked to use CrashSimulator to test
%the same type of applications evaluated in our initial tests.
%The results show that the developers were able to find bugs
%that were missed by the applications' existing testing strategies.
%Additionally, the
%participants were able to find bugs
%in environments with which they had only a limited amount of experience.

The main contributions in this work can be summarized as follows:

\begin{itemize}

\item{An expanded version of the Simulating Environmental Anomalies (SEA)
  technique that can expose bugs both between an application and its
    environment and between application components that communicate with
    one another}

\item{It demonstrates a new language, {\em CSLang},
which allows for concise descriptions of transducers that may
modify streams of communication between components in order to simulate
anomalous conditions}

\item{It proves the value of CSLang by illustrating the ease with which
  transducers can be constructed and the effectiveness with which they are
    able to find bugs in real world applications }

\end{itemize}
