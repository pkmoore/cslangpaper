\section{Evaluation}
\label{SEC:evaluation}

With our prototype in hand, we designed a set of experiments to evaluate
whether PORT would be effective in real world situations.
Specifically, we aim to answer the following questions:

\begin{itemize}

    % Re-implementing CS anomalies
  \item{Can PORT express the anomalies used by SEA to identify bugs?}

    % Extending to support RPC formats
  \item{How easy is it to extend PORT to support activity representations
    other than system calls?}

    % Dornhackl et al. defining maliciousness
  \item{What problems can be addressed by employing PORT on
  non-system-call activity representations?}

    % Performance information
  \item{Can PORT process input streams in a reasonable amount of time?}

\end{itemize}


\subsection{Expressing SEA Anomalies}
\label{sub:SEAAnomalies}
Given that this work is motivated
in large part
by the effectiveness of the SEA technique,
our first experiment aims to reproduce the anomalies described
in~\cite{DBLP:conf/issre/MooreCFW19}.
Specifically,
we test PORT's ability to recreate
the study's unusual file type mutator,
and its cross-disk file move checkers, which were used to identify
the bulk of the bugs that were found.
% Our hope was that by using PORT we could eliminate boiler plate code,
% save effort by handling common tasks automatically, and improve reliability
% by providing a structured way to modify and output system call sequences.



\subsubsection{Creating the Unusual File type Mutator}
\label{subsub:UnusualFiletype}
For the first part of this experiment,
we used PORT to implement an ``unusual file type''
mutator.
As illustrated in Figure~\ref{lst:SEAListings},this mutator
takes an input trace
that contains a call to either {\tt stat()},
{\tt fstat()},
or {\tt lstat()}
and modifies the call's result data structure such
that its {\tt ST\_MODE} member will contain a value
that indicates an unusual file type.
As can be seen in
Figure~\ref{lst:SEAListings}, this task can be expressed with only a few lines of PORT code.  In the figure,
lines 1 through 4 define what {\tt stat()}, {\tt fstat()}, and {\tt
lstat()} calls look like, and which parameter contains the result buffer.
Line 6 generates an accepting state that, when entered, produces an output
system call with a modified value in the return structure's {\tt st\_mode}
field.  The output can then be used to modify the results of a running
application's system calls in order to carry out the remaining steps of the
SEA technique.

\begin{figure}
\centering
\begin{minipage}[t]{.5\textwidth}
\begin{lstlisting}[basicstyle=\ttfamily\scriptsize]
event fstat {filedesc: Numeric@0};
event stat {filename: String@0};
event open {filename: String@0, filedesc: Numeric@ret};

stat({filename: fn});
open({filename: ?fn, filedesc: fd});
fstat({fildesc: ?fd});
\end{lstlisting}
\end{minipage}%
\begin{minipage}[t]{.5\textwidth}
\begin{lstlisting}[basicstyle=\ttfamily\scriptsize]
event Statbuf {mode: String@2};
event anystat {stat sb: Statbuf@1}
        | {lstat sb: Statbuf@1} 
        | {fstat sb: Statbuf@1};
anystat({sb: {mode: -> "st_mode=S_IFBLK"}});
\end{lstlisting}
\end{minipage}
\caption{The left listing contains a
PORT program that can detect whether a file was replaced as it was copied.
The right listing shows a program that
identifies a \texttt{stat}, \texttt{lstat}, or \texttt{fstat} call and modifies
  the \lstinline+ST\_MODE+ member of its \lstinline+statbuf+ output parameter to contain the value
  \lstinline+"S\_IFBLK"+. This indicates that the file being examined is a block device rather than a regular file.}
\label{lst:SEAListings}
\end{figure}

The original implementation of this mutator in~\cite{DBLP:conf/issre/MooreCFW19} consisted of 55 lines
of Python code.
When compared to the original mutator, it became apparent that there were
several major advantages
to using
PORT:

\textit{Minimal boilerplate code:} The PORT program lacks the boilerplate
code associated with
reading an input trace, managing mutator state, and producing output.
This is possible because PORT's capabilities are narrowly defined to
only describe the states and operations of a transducer.  As a result,
functions can be generically implemented within PORT's core, eliminating
the need for users to do so manually.

\textit{No code required to filter out uninteresting calls:}
In PORT, there is no
need to explicitly exclude system
calls outside of the desired set.  Each statement defines a new state with
incoming and outgoing transitions configured such that any system calls not
dealt with in the PORT program are ignored.

\textit{Easy to modify call contents:}  PORT's operators make it
easy to change only those parameters  of a system call
 needed to produce output.
This is a far cry
from the Python program, which relies on manual and fragile string manipulation
to achieve the same effect.

\subsubsection{Supporting Cross-Disk Move Checkers}

In the second part of this experiment we test whether PORT can be used to
implement the ``checkers'' used in the SEA work to determine if an
application can correctly move a file from one disk to another.
This task is a common source of bugs in Linux applications. As the
Linux
{\tt rename()} system call does not support moving files from one disk to
another,
applications must perform this complex
operation themselves.
Moore et al. identified the steps required to
correctly perform such a move by examining the source code of the ``mv''
command and implemented a set of checkers to identify situations where one
of these steps is not carried out correctly by an application.
In real world applications,
these checkers were able to identify bugs
in many popular applications and libraries that offer file movement
capabilities.

We evaluated each of the four checkers listed in Moore et al.'s work and
determined that PORT could be used to implement three of them.
Figure~\ref{lst:SEAListings} shows our PORT implementation of
the ``File Replaced During Copy'' checker
that ensures {\tt fstat} is used after a file is opened but
before it is moved.  This pattern indicates that an application may be
storing the {\tt inode} number of the file -- a step necessary to prevent a race
condition where the file is replaced during the move process.
Comparing it to the
original SEA checker,
which consisted of 45 lines of Python,
shows that many of the same deficiencies that affected the unusual file type
mutator above are present.
The PORT version is much more concise and its meaning is
less obscured by boilerplate and state management code that makes up the
bulk of the Python version.

This exercise did expose one of PORT's shortcomings.  Specifically,
we found that PORT cannot currently implement the ``Extended File
Attributes'' checker, which
 ensures that an application
preserves all of a file's extended attributes and re-applies them after the move.
PORT's lack of a list data structure made it difficult to create this checker as a
list is required to capture the values {\tt getxattr()}
and ensure they have all been applied with a
corresponding call to {\tt setxattr()}.
Though we are considering such a feature for future implementation,
we do not currently support it because such an extension could
hurt program clarity and make it more difficult to reason about
mutator behavior.


\subsection{Extending PORT to Other Activity Representations}

Our second experiment evaluates PORT's usability beyond system call manipulation.
A logical next step is to test PORT's capabilities when working
with a higher level, but similarly structured, activity representation, such as remote procedure calls.
We chose the JSONRPC and XMLRPC APIs for this purpose as they are widely used and have well
supported parsing libraries or modules.

To evaluate how well PORT can work with these representations we
construct transformer modules that convert these formats into our
intermediate data format.
We test our implementation by writing PORT programs that modify
activity streams similar to the examples presented in the JSONRPC
2.0~\cite{jsonspec} and XMLRPC~\cite{xmlspec}
specifications.  One such program is shown in
Figure~\ref{lst:JSONProgram}.  This program matches a pattern of ``test''
and ``update'' calls and, if the pattern is found, the final update call's
parameters are replaced with the constants 999 and 888, respectively.

Adding support for JSONRPC and XMLRPC by creating
appropriate transformers required only a handful of hours of effort.
Based on the ease and speed with which we
were able to complete these new input stream formats, we are confident that PORT can be
quickly adapted to support new activity representations as required.

\begin{figure}[t]
\centering
\begin{tabular}{c}
\begin{lstlisting}
event update {up1: Number@0, up2: Number@1};
event test {tp1: Number@0, tp2: String@1};
update({up1: pone, up2: ptwo});
test({tp1: 45, tp2: "alpha"});
update({up1: ->999, up2: ->888});
\end{lstlisting}
\end{tabular}
\caption{This PORT program matches a pattern of JSONRPC calls to
  ``update'' and ``test.''  If the pattern is identified, the final call to
  update is modified so that its first two parameter values are replaced
  with new constant values.}
\label{lst:JSONProgram}
\end{figure}


\subsection{Utilizing PORT's Flexibility}
We further evaluate PORT's ability to replicate useful strategies by
reproducing Dornhackl et al.'s~\cite{Dornhackl2014}
efforts to describe and detect malicious behavior in Windows
applications.  The Dornhackl team was able to describe what it takes to implement a malicious goal, and then
map those tasks onto the
concrete Windows API calls required to carry them out. PORT accomplishes the task by making one operation out of a set of similar API calls and then describing a malicious task  in terms of
these aggregate operations. As illustrated in  Figure~\ref{lst:DornhacklOpen} PORT uses variants to group the API calls that may be used to
open a Windows registry key \cite{Dornhackl2014}. 

\begin{figure}[t]
\begin{lstlisting}
OPEN => RegOpenKeyA NtOpenKey
  | RegOpenKeyW NtOpenKey
  | RegOpenKeyExA NtOpenKey
  | RegOpenKeyExW NtOpenKey
\end{lstlisting}
\caption{Grouping of Windows API calls for opening or creating a Windows
  registry key into an OPEN operation as per Dornhackl et al.}
\label{lst:DornhacklOpen}
\end{figure}

Each symbol, beginning with ``Reg'' in Figure~\ref{lst:DornhacklOpen}, represents a user
API call that opens or creates a registry key.  {\tt NtOpenKey} always follows
these calls because it is the underlying ``native'' call that actually performs the operation.
Figure~\ref{lst:PORTOpenReg} shows the grouping and execution semantics as expressed
in PORT.

\begin{figure}[t]
\begin{lstlisting}[gobble=2]
  event open {RegOpenKeyA ...} | {RegOpenKeyW ...};
    # ... further variants omitted
  event NtOpenKey {...};

  open({...});
  NtOpenKey({...});
\end{lstlisting}
  \caption{Abstract PORT program (with parameters
  omitted) that groups Windows API calls responsible for opening or
  creating a Windows registry key.  It also shows
  how the requirement that \lstinline+NtOpenKey+ follow any of these calls can be
  captured.}
\label{lst:PORTOpenReg}
\end{figure}

Though they use different approaches to do so,  both Dornhackl et al. and PORT can describe a pattern that, if
detected, indicates that
a Windows registry key has been installed
that can cause some malicious action
the next time the machine is restarted.
% (assuming PORT were extended to support Windows).
The PORT program is shown in Figure~\ref{lst:PORTRegDetect}.  The program groups API calls into \lstinline+OPEN+, \lstinline+SET+, and \lstinline+CLOSE+ operations and searches for a pattern that
indicates
an application is
setting an autostart key.  PORT can also express
that the pattern must appear for a specific registry key.
%using the value stored in the ``regkey'' register.

\begin{figure}[t]
\centering
\begin{tabular}{c}
\begin{lstlisting}[gobble=2]
  event open {RegOpenKeyA ...} | {RegOpenKeyW ...} | ...;
    # further variants omitted
  event NtOpenKey {...};
  event set {RegSetValueExA ...} | {RegSetValueExW ...}
            | ...;  # further variants omitted
  event NtSetValueKey {...};
  event close {RegCloseKey ...};
  event NtClose {...};

  open({...});
  NtOpenKey({...});
  set({...});
  NtSetValueKey({...});
  close({...});
  NtClose({...});
\end{lstlisting}
\end{tabular}
  \caption{This listing shows an abstract PORT program (with parameters
  omitted) that detects situations where an application is maliciously
  installing an ``autostart'' Windows registry key.  It does so by
  implementing the pattern described by Dornhackl et al.  Unimportant
  parameters are omitted and some API call variants in each group have been
  eluded.}
\label{lst:PORTRegDetect}
\end{figure}


\subsection{PORT's Performance}

%It doesn't matter how useful a tool may be
%if it takes too long to complete its work.
%Though our implementation is
%only a prototype, we wanted to make sure that its performance was not
%overly slow.
Our final experiment evaluates the
time required for our prototype implementation
to identify specific
patterns within real-world system call traces.
We chose two widely used network applications that offered a sufficient level of complexity,
NCat
and
Python's http server,
and recorded test traces using the following 
experimental setup.  The applications are configured to service
a simple piece of content (a single string in the case of NCat), and their system call activity is
recorded using {\tt strace} while handling a request from a remote client.
The strace recordings are then processed using the PORT program shown in
Figure~\ref{lst:RealWorldPerformance},  which
identifies the sequence of system calls that implement
a server's request handling loop\footnote{Recordings are pre-processed to remove system calls
related to executable loading and process creation.}.  Table~\ref{tbl:RealWorldPerformance}
shows the times in seconds required to perform this identification on each
web server, as well as the total number of system calls in each trace.

\begin{figure}[t]
\centering
  \begin{tabular}{|c|l|l|l|l}
                & 1 Execution & 10 Executions & 100 Executions & Num. Syscalls\\
              \hline
  rsync client  & 0.128       & 1.257         & 12.067         & 274 \\
  ftp client    & 0.200       & 1.858         & 18.616         & 891 \\
  scp client    & 0.172       & 1.756         & 17.262         & 490 \\
  ssh client    & 0.232       & 2.273         & 22.546         & 850 \\
  telnet client & 0.114       & 0.958         & 9.132          & 23  \\
  http.server   & 0.104       & 0.914         & 8.823          & 43  \\
  ncat server   & 0.115       & 0.933         & 9.140          & 43  \\
  socat server  & 0.118       & 1.124         & 10.321         & 71  \\
\end{tabular}
\caption{Time in seconds to process the listed number of events.  The 2nd - 4th
columns list times required to complete 1, 10, and 100 executions.  Column 5
notes the number of system calls in the trace being processed.}
\label{tbl:RealWorldPerformance}
\end{figure}

%\begin{figure}
%\centering
%\begin{tabular}{c}
%\begin{lstlisting}[gobble=2]
%  event accept { accept fd: Number@ret}
%             | { accept4 fd: Number@ret};
%  event anyrecv { recvfrom fd: Number@0}
%    | { read fd: Number@0} | { recv fd: Number@0};
%  event anysend {sendto fd: Number@0}
%    | { write fd: Number@0} | { send fd: Number@0};
%  event close {fd: Number@0};
%
%  accept({fd: storefd});
%  anyrecv({fd: ?storefd});
%  anysend({fd: ?storefd});
%  close({fd: ?storefd});
%\end{lstlisting}
%\end{tabular}
%\caption{A PORT program that matches patterns where a server application
%  accepts a connection, receives a request, sends a response, and closes
%  the connection.  The program uses variants to handle cases where
%  applications use different system calls to perform some common action
%  (e.g. receiving data from a socket).}
%\label{lst:RealWorldPerformance}
%\end{figure}

The results in Table~\ref{tbl:RealWorldPerformance} show that
PORT's
processing time increases in line with the total number of system calls
in the recording.  We anticipate that much of this processing cost is
associated with setting up the Python execution environment and that a more
optimized implementation could improve performance gains in this area.
Further,
it is likely that PORT's performance is closely tied to
disk throughput,
and that advancing the transducer
as each system call is evaluated
adds little additional overhead.
%A condition with varying disk speeds could be designed to confirm this suspicion.

%%% Local Variables:
%%% mode: latex
%%% TeX-master: "paper"
%%% End:
